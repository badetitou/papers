\documentclass[conference]{IEEEtran}
\usepackage[T1]{fontenc} %%%key to get copy and paste for the code!
\usepackage[utf8]{inputenc} %%% to support copy and paste with accents for french stuff
\usepackage{times}
\usepackage[scaled=0.85]{helvet}
\usepackage{graphicx}
\usepackage{ifthen}
\usepackage{xspace}
\usepackage{alltt}
\usepackage{latexsym}
\usepackage{url}
\usepackage{amsmath,amssymb,amsfonts}
\usepackage{stmaryrd}
\usepackage{algorithmic}
\usepackage{textcomp}
\usepackage{xcolor}
\usepackage{enumerate}
\usepackage{cite}
\usepackage[pdftex,colorlinks=true,pdfstartview=FitV,linkcolor=blue,citecolor=blue,urlcolor=blue]{hyperref}
\usepackage{xspace}


\author{
        \IEEEauthorblockN{Abderrahmane Seriai, Laurent Deruelle}\IEEEauthorblockA{Berger-Levrault, France} 
        \and
	    \IEEEauthorblockN{Beno\^{i}t Verhaeghe, Anne Etien, \\ Nicolas Anquetil, St\'{e}phane Ducasse}\IEEEauthorblockA{Universit\'{e} de Lille, CNRS, Inria, \\ Centrale Lille, UMR 9189 -- CRIStAL, France\\}
}

\begin{document}
\title{GWT to Angular: Migration of graphic applications through Models}
% Si méthode autre fonctionne... Application de la méthode à notre cas d'étude ...

\date{\today}
\maketitle



\begin{abstract}
    In the context of the graphical application evolution, it happens that the application must switch of implementation langage.
    Berger-Levrault is a major IT company witch owns the biggest GWT application in the world. 
    The company needs to switch the application implementation of all its software from GWT to Angular.
    Rewriting graphical application is complex and can lead to errors.
    The company has estimated the duration of the migration to .
    To reduce the among of time needed, we've created a tool that semi-automatically do the migration of the application.
    To achieve this tool, we've designed a meta model to represent a graphical user interface.
    Then we've created a three parts application migration strategy and the tools to apply such strategy.
    Firstly, we've instantiated a meta model from the source code.
    Then, we've generated a model that respects the meta model we've defined to represent a graphic application from the model of the source code.
    Finally, we've exported the \textit{"graphic"} model to the target code.
\end{abstract}

\begin{IEEEkeywords}
    Graphical User Interfaces, Industrial case study, Java, Angular, GWT
\end{IEEEkeywords}


\section{Introduction}
\label{sec:intro}

% Contexte
% - Entreprise internationale
% - Logiciel de plus de X MLOC
% - Nombre d'écran
% - Année d'existence (RH +~ 10 ans)
% - Nombre de dev
% - GWT en perte de vitesse

Berger-Levrault is a major IT company.
It uses the framework GWT to develop its graphics applications.
GWT doesn't evolve these years with only one major release since 2015 whereas there was six major release between 2010 and 2013.
So, Berger-Levrault has decided to migrate all its application from Java using GWT to the Angular technology.
The applications Berger-Levrault are long life applications (more than ten years),
    they have more than X MLOC and represent
    X web pages. 
This specificities plus the fact that developers don't work in the same company for years implies
    a problem of perfect knowledge of the applications.
No one know all the code of an application and during a migration so
    it leads to migration of unused code or \emph{hack} important in the source language but witch leads to error in the target one.
This is why such migration is time consuming and error prone.
Berger-Levrault has estimated the duration of the migration to
At the same time, the company needs to continue the development of their applications during the migration.

% Problème
% - avoir un outil de représentation de haut niveau pour des applications graphique
% - ne migrer que les éléments essentielles (suppression du code mort)
% - Target independence
% - Changement de langage imposant un changement dans l'architecture du projet et la division en plusieurs fichiers.
% -- Division en plusieurs fichier -- Complex interface mais pas d'analyse dynamique 

The migration of an application is not only is the transformation of the source code to the target language. 
It also implies to find the frameworks to use in the target language 
    and to think about the architecture of the applications in the new language.
The migration should have a really tiny impact for the client of the company and for the developers of the application.
In the case of Berger-Levrault, multiple applications are developed following the same architecture
    but some other used different framework or language like ReactJS, Aurelia, Laravel or Silverlight.
So, it is important to have a solution that can be reused for all the graphic applications with a minimum of modification.

% Known tracks for \sd{solutions} here you want to show that you are not an idiot not knowing what have been around
% Pourquoi on ne peut pas utiliser les autres papiers ( du point de vue leurs methodologies )
% - Stratégie dynamique inexploitable au vue de la combinatoire induit par la taille application, et ne peut pas être mise chez les clients (trop lente) -- \cite{samir2007swing2script}
% - Pour RAD \cite{sanchez2014model}, la complexité des interfaces de Berger-levrault fait que des éléments ne sont pas visible à l'écran

To extract information from the application, it is possible to use a dynamic solution or static one. 

\cite{samir2007swing2script} have proposed a dynamic solution to migrate Swing application to Ajax.
Their tool run an instance of the source application in a browser.
Then, they are able to detect the widget displayed by the interface and the different actions available.
The authors have decided to use a dynamic solution to explore the interface of the application.
Their contributions are a method and a middleware toolkit for re-architecting Swing application to Ajax one.

\cite{sanchez2014model} have developed a static solution to reverse engineer RAD (Rapid Application Development) based graphic user interface (GUI).
The authors have created different meta model to represent a GUI.
The authors have used those to meta models to create a chain of transformation between them witch lead to the transformation of 
    the source  application the target one.
Although we can't reapply the strategy of GUI extraction to our project, because RAD based application are too \emph{simple} compare to GWT based application,
    the different meta models and the chain of transformation inspired our work.

% What our solution is \ct{Set} and \ct{OrderedCollection} (so that the reader knows where the paper is going)
% - Un outil réutilisable
% - Migrer l'ensemble du code lié à l'interface Graphique (sans code métier) -> Mesurable
% - Recherche de correspondance entre ancien et nouveau framework (GWT to HTML/PrimeNG)

We present a new static solution to represent a GUI developed with Java-Swing, 
    a strategy to generate this model,
    a way to generate the target application whatever the target framework Architecture.
Our solution is source and target independent.
The tool detect well X\% of the widget for a web page.
After the migration, X\% widgets are well placed and X\% are visually identical.

% Contribution of the paper
% - Une méthodologie et un outil modulaire pour migrer l'aspect graphique d'une application

The main contributions of our work are:
\begin{itemize}

    \item a modular tool that can be reuse for other graphic application migration.

    \item a meta model to represent a GUI application.

    \item a migration strategy to migrate graphic application.
    
\end{itemize}

% Paper structure

In Section~\ref{sec:problem}, we give background information on our problem. 
Then, in Section~\ref{sec:contribution}, we describe the solution we've proposed.
Section~\ref{sec:discussion} presents the results and threats to validity of our works.
Finally, we present the related works and conclude in Section~\ref{sec:related} and~\ref{sec:conclusion} respectively.

\section{Problem Description}
\label{sec:problem}

% Context, exposed with the \textbf{most precise terms possible} (don't open
% unwanted doors for the reader)

Berger-Levrault is a major IT company witch needs to migrate
    their GWT-based applications to Angular.
The migration of those applications will lead to X days of development.
This is due to the X MLOC that composed the software they have to migrate.
Our work was tested on the showroom application of Berger-Levrault. 
This application is used only by the company's developers as
    an example application of the usage of the widget available for the development.

% Probably set the vocabulary before to cut any misinterpretation

% Constraints that influenced the solution (because the solution is not
% universal) \emph{e.g.} our requirements for a solution, possibly not all
% satisfied. They should be sound and believable. Analysis of the criteria.
% Imagine that you are another guy having this problem do the constraint
% matches yours so that you could apply the solution


The solution we propose must respect some criteria:
\begin{itemize}
    \item \emph{Source independence}. Our solution should be easy to adapt
        for all project written in GWT using Java and for
        application not written in Java but describing a GUI.
    \item \emph{Target independence}. It has to be easy to change the target
        language without restructuring the meta model we have defined or 
        change any stage of the migration.
    \item \emph{Modularity}. The migration would be split into many little
        stages. This offer the possibility to easily extend on step or 
        replace one step by another one without introducing instability. 
    \item \emph{Preserving Architecture}. After the migration, we should find
        the same architecture between the different component of the GUI (\emph{e.g.}
        a button witch belongs to a panel in the source application still belongs to the same panel in the target application ).
    \item \emph{Layout-preserving visual}. It should not have visual difference
        between the source application and the target one.
\end{itemize}

The development team have to continue the maintenance of the source application.
This is a constraint inherent to industrial company.

% Problem

% Factual solution tracks, to position...
% Our solution in a nutshell.


\section{Proposed Solution}
\label{sec:contribution}

\subsection{Meta model}

\subsection{Import}

\subsection{Export}

% Free form, variable number of sections, technical details.

% But in general do not mix solution and discussions/possible variation
% let that for discussion

\section{Discussion}
\label{sec:discussion}

% Critères d'évaluation
% - Ne pas déstabiliser les développeurs (garder l'architecture du logiciel proche) - évaluation avec les développeurs de la solution qu'on leur propose
% - Nombre de phase correctement détecté
% - Nombre de widget correctement détecté (on peut compter à la main sur 20 pages web)
% - Nombre de widget correctement placé (dans l'architecture sûrement pas 100% mais pas mal)
% - Nombre de widget visuellement correctement placé (notion d'attribut, de panel).. Ici on aura de mauvais résultat qui proviennent de l'analyse statique et non dynamique (qui est une contrainte forte pour nous)

% Discussion of actual solution \emph{vs.} initial constraints from

% Evaluation of the solution. How does the solution meet the criteria? Where
% does it succeed or fails...


\section{Related Works}
\label{sec:related}

% Other solutions in the domain, and a real comparison of our contribution with
% solutions from other people.


\section{Future Works}
\label{sec:future}

% Futur Work
% - Détection du Code Metier
% - Préservation de la disposition des éléments graphique dans l'interface (layout)
% - Préservation de l'aspect graphique des éléments graphiques (css)

\section{Conclusion}
\label{sec:conclusion}

% In this paper, we \textsf{looked}\xspace at problem P with this context and these
% constraints. We proposed solution S. It has such good points and such not so
% good ones. Now we could do this or that.

\subsection*{Acknowledgements} 
This work was supported by Ministry of Higher Education and Research, Nord-Pas de Calais Regional Council, CPER Nord-Pas de Calais/FEDER DATA Advanced data science and technologies 2015-2020.

\bibliographystyle{abbrv}
\bibliography{references}

\end{document}