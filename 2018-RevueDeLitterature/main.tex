\documentclass[conference]{IEEEtran}
\usepackage[T1]{fontenc} %%%key to get copy and paste for the code!
\usepackage[utf8]{inputenc} %%% to support copy and paste with accents for french stuff
\usepackage{times}
\usepackage[scaled=0.85]{helvet}
\usepackage{graphicx}
\usepackage{ifthen}
\usepackage{xspace}
\usepackage{alltt}
\usepackage{latexsym}
\usepackage{url}
\usepackage{amsmath,amssymb,amsfonts}
\usepackage{stmaryrd}
\usepackage{algorithmic}
\usepackage{textcomp}
\usepackage{xcolor}
\usepackage{enumerate}
\usepackage{cite}
\usepackage[pdftex,colorlinks=true,pdfstartview=FitV,linkcolor=blue,citecolor=blue,urlcolor=blue]{hyperref}
\usepackage{xspace}


% natbib
\usepackage[numbers]{natbib}
\input{macros}


\author{
    \IEEEauthorblockN{Beno\^{i}t Verhaeghe$^{1,2}$, Anne Etien$^1$,\\ Nicolas Anquetil$^1$, St\'{e}phane Ducasse$^1$}\IEEEauthorblockA{$^1$Universit\'{e} de Lille, CNRS, Inria, \\ Centrale Lille, UMR 9189 -- CRIStAL, France\\}
    \and
    \IEEEauthorblockN{Abderrahmane Seriai$^2$, Laurent Deruelle$^2$,\\ Mustapha Derras$^2$}\IEEEauthorblockA{$^2$Berger-Levrault, France}     
}

\begin{document}
\title{GUI application meta-models\\ State of the Art}


\date{\today}
\maketitle



\begin{abstract}

\bvc{In this context...}
When a developer wants to analyse an application,
    and this application includes an user interface.
He could want an abstraction of the application.
Often, this abstraction level correspond to models, and their meta-models, of
    the piece of software.
\bvc{We consider this problem P...}
There are indeed many elements to represent and to link.
\bvc{P is a problem because...}
Currently there are many meta-models of GUI application
    that can be used to represent the interface
    and some links between the different windows.
But none of them express complex behavior, such as loop or condition,
    nor data structure information.
\bvc{We propose this solution...} 
We defined four meta-models.
The first one represent the Graphical User Interface,
    the second one the layout to apply to the GUI, 
    the third one the data structure implies in the GUI, 
    the last one the behavior associate to an event fired by an element of the GUI. 
\bvc{Our solution solves P in such and such way.}
Our meta-models can express the different elements of a GUI application.
So can represent the graphical user interface
    and the logic of the application.

   
\end{abstract}

\begin{IEEEkeywords}
    Graphical User Interfaces, Model-Driven Engineering
\end{IEEEkeywords}

\section{Introduction}
\label{sec:intro}

% Contexte
\bvc{Contexte} \bvc{use case}
In the context of the analysis of an application.
It happens that the developers want to create tools on top of his analysis.
These tools could be useful in cases like:
    analysis, tests generation, migration, \etc.

\bvc{introduction model}
A way to create those software is the usage of model of the source code of the application.
The developers create or use a meta-model of the language of the application source code.
Then they instantiate this meta-model from the application to analysed.

% Problème
\bvc{Problème}
\bvc{intro gui}
In the case of GUI application, it happens that the abstraction level doesn't provide enough information.
Indeed, the generated model contains the methods, classes, \etc. of the source application,
    but no information about how the GUI is shaped.
The developers must do another analysis on the model to extract these information and so
    making his tool.

\bvc{intro simple gui decomposition}
A solution to create tool specialized for GUI application is to create GUI model.
A GUI application is divided into different elements.
The aim of this paper is to define those components and meta-models
    to represent all the specificities linked to a GUI application.

% Known tracks for \sd{solutions} here you want to show that you are not an idiot not knowing what have been around
\bvc{Know tracks}
We did not find any other papers which defines a solution to represent graphical application.
Nevertheless, the KDM model designed by the OMG proposed a \textit{Resource Layer} which can used to define an GUI application.
Their solution is discussed section X.

% What our solution is \ct{Set} and \ct{OrderedCollection} (so that the reader knows where the paper is going)
\bvc{What is our solution}
We defined four meta-models to represent the GUI software.
The meta-models represent the different main GUI's specificities we extracted from our analysis
    and other research papers.

% Contribution of the paper
\bvc{Contrib of the paper}
The main contributions of our work are: 

\begin{itemize}
    
    \item Description of GUI application structure

    \item Meta-Models to represent a GUI application

    \item Discussion about GUI Meta-Models

\end{itemize}

% Paper structure
\bvc{Paper structure}
In Section~\ref{sec:guiAppDiv}, we present the different GUI elements. 
Then, in Section~\ref{sec:solutions}, we describe the solution we've proposed.
Section~\ref{sec:contribution} exposes our solution.
Finally, we conclude in Section~~\ref{sec:conclusion}.


\section{GUI application structure}
\label{sec:guiAppDiv}


\section{Existing solutions}
\label{sec:solutions}

% Context, exposed with the \textbf{most precise terms possible} (don't open
% unwanted doors for the reader)


% Probably set the vocabulary before to cut any misinterpretation

% Constraints that influenced the solution (because the solution is not
% universal) \emph{e.g.} our requirements for a solution, possibly not all
% satisfied. They should be sound and believable. Analysis of the criteria.
% Imagine that you are another guy having this problem do the constraint
% matches yours so that you could apply the solution


% Factual solution tracks, to position...
% Our solution in a nutshell.

\section{Proposed meta-models}
\label{sec:contribution}

% Free form, variable number of sections, technical details.

% But in general do not mix solution and discussions/possible variation
% let that for discussion



\section{Conclusion}
\label{sec:conclusion}

% In this paper, we \textsf{looked}\xspace at problem P with this context and these
% constraints. We proposed solution S. It has such good points and such not so
% good ones. Now we could do this or that.

In this paper, we defined four meta-models to represent a GUI applications.

The \textbf{GUI model} represents the different \textit{pages} of the applications and their contents.
It includes the properties of the widgets and the link from the widgets and their events.

The \textbf{layout model} expresses the positioning relation between the widgets of the GUI models.

The \textbf{behavior model} defines the behavior to execute when
    an event is fired.
The events are fired by using an action on a widget of the GUI model or
    automatically by the application.

The \textbf{data model} includes information about the data manipulated
    by the GUI application.
Those data can be used by the behavior model and transmit to the
    GUI model.

\subsection*{Acknowledgements} 
This work was supported by Ministry of Higher Education and Research, Nord-Pas de Calais Regional Council, CPER Nord-Pas de Calais/FEDER DATA Advanced data science and technologies 2015-2020.

%\bibliographystyle{abbrv}
%natbib 
\bibliographystyle{myplainnat}
%\bibliography{references}

\end{document}