\documentclass[conference]{IEEEtran}
\usepackage[T1]{fontenc} %%%key to get copy and paste for the code!
\usepackage[utf8]{inputenc} %%% to support copy and paste with accents for french stuff
\usepackage{times}
\usepackage[scaled=0.85]{helvet}
\usepackage{graphicx}
\usepackage{ifthen}
\usepackage{xspace}
\usepackage{alltt}
\usepackage{latexsym}
\usepackage{url}
\usepackage{amsmath,amssymb,amsfonts}
\usepackage{stmaryrd}
\usepackage{algorithmic}
\usepackage{textcomp}
\usepackage{xcolor}
\usepackage{enumerate}
\usepackage{cite}
\usepackage[pdftex,colorlinks=true,pdfstartview=FitV,linkcolor=blue,citecolor=blue,urlcolor=blue]{hyperref}
\usepackage{xspace}


\author{
    \IEEEauthorblockN{Beno\^{i}t Verhaeghe, Anne Etien, \\ Nicolas Anquetil, St\'{e}phane Ducasse}\IEEEauthorblockA{Universit\'{e} de Lille, CNRS, Inria, \\ Centrale Lille, UMR 9189 -- CRIStAL, France\\}
    \and
    \IEEEauthorblockN{Abderrahmane Seriai, Laurent Deruelle}\IEEEauthorblockA{Berger-Levrault, France}     
}

\begin{document}
\title{How to migrate GUI interface?}

% Planning the review
% - Need for a review 
% - Commissioning a review
% - Specifying the research Question
% - Developing a review protocol
% - Evaluating the review protocol
%
% Conducting the review
% - Identification of research
% - Primary studies
% - Study quality assessment
% - Data extraction and monitoring
% - Data synthesis
%
% Reporting the review
% - Specifying dissemination mechanisms
% - Formatting the main report
% - Evaluating the report


\date{\today}
\maketitle



\begin{abstract}

% In this context...
% We consider this problem P...
% P is a problem because...
% We propose this solution...
% Our solution solves P in such and such way.

Avec l'évolution des solutions vers le web, de plus en plus d'entreprise souhaite migrer leurs application \textit{desktop} vers 
    des application \textit{web}.
De manière plus générale
   
\end{abstract}


\section{Introduction}
\label{sec:intro}

% Contexte
% - Entreprise internationale
% - Logiciel de plus de X MLOC
% - Nombre d'écran
% - Année d'existence (RH +~ 10 ans)
% - Nombre de dev
% - GWT en perte de vitesse


% Problème


% Known tracks for \sd{solutions} here you want to show that you are not an idiot not knowing what have been around


% What our solution is \ct{Set} and \ct{OrderedCollection} (so that the reader knows where the paper is going)
% - Un outil réutilisable
% - Migrer l'ensemble du code lié à l'interface Graphique (sans code métier) -> Mesurable
% - Recherche de correspondance entre ancien et nouveau framework (GWT to HTML/PrimeNG)

% Contribution of the paper


% Paper structure


\section{Problem Description}
\label{sec:problem}

Nous avons trouvé plusieurs technique pour effectuer la migration. 

\begin{itemize}

    \item \textit{Manuellement}. Une petite description ici

    \item \textit{Rule engine}. Une petite description ici

    \item \textit{Through Model}. Une petite description ici

\end{itemize}

Pour chaque cas que nous avons trouvé, nous avons défini les critères suivants : 

\begin{itemize}

    \item \textit{Lisibility}. Description

    \item \textit{Compilable}. Description

    \item \textit{Conservation architecture source}.

    \item \textit{Respect architecture cible}.

    \item \textit{Couverture}.

    \item \textit{Source et cible indépendance}.

    \item \textit{Modulaire}.

    \item \textit{Automatique}.

    \item \textit{Amélioration du code source}.

    \item \textit{Préservation de l'aspect}.

\end{itemize}


\section{Migration Solution}
\label{sec:Solution}

\subsection{Manuellement}
\label{sec:Manually}

\subsection{Rule Engine}
\label{sec:ruleEngine}

\subsection*{Normalisation}

\subsection{Through Model}
\label{sec:throughModel}

\subsection*{Dynamic}

- Crawling 

- ? 

\subsection*{Static}

% Context, exposed with the \textbf{most precise terms possible} (don't open
% unwanted doors for the reader)


% Probably set the vocabulary before to cut any misinterpretation

% Constraints that influenced the solution (because the solution is not
% universal) \emph{e.g.} our requirements for a solution, possibly not all
% satisfied. They should be sound and believable. Analysis of the criteria.
% Imagine that you are another guy having this problem do the constraint
% matches yours so that you could apply the solution


% Factual solution tracks, to position...
% Our solution in a nutshell.

\section{Proposed Solution}
\label{sec:contribution}

% Free form, variable number of sections, technical details.

% But in general do not mix solution and discussions/possible variation
% let that for discussion



\section{Conclusion}
\label{sec:conclusion}

% In this paper, we \textsf{looked}\xspace at problem P with this context and these
% constraints. We proposed solution S. It has such good points and such not so
% good ones. Now we could do this or that.

 

\subsection*{Acknowledgements} 
This work was supported by Ministry of Higher Education and Research, Nord-Pas de Calais Regional Council, CPER Nord-Pas de Calais/FEDER DATA Advanced data science and technologies 2015-2020.

\bibliographystyle{abbrv}
% \bibliography{references}

\end{document}