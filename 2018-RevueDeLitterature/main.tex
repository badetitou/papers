\documentclass[conference]{IEEEtran}
\usepackage[T1]{fontenc} %%%key to get copy and paste for the code!
\usepackage[utf8]{inputenc} %%% to support copy and paste with accents for french stuff
\usepackage{times}
\usepackage[scaled=0.85]{helvet}
\usepackage{graphicx}
\usepackage{ifthen}
\usepackage{xspace}
\usepackage{alltt}
\usepackage{latexsym}
\usepackage{url}
\usepackage{amsmath,amssymb,amsfonts}
\usepackage{stmaryrd}
\usepackage{algorithmic}
\usepackage{textcomp}
\usepackage{xcolor}
\usepackage{enumerate}
\usepackage{cite}
\usepackage[pdftex,colorlinks=true,pdfstartview=FitV,linkcolor=blue,citecolor=blue,urlcolor=blue]{hyperref}
\usepackage{xspace}


% natbib
\usepackage[numbers]{natbib}
\input{macros}


\author{
    \IEEEauthorblockN{Beno\^{i}t Verhaeghe$^{1,2}$, Anne Etien$^1$,\\ Nicolas Anquetil$^1$, St\'{e}phane Ducasse$^1$}\IEEEauthorblockA{$^1$Universit\'{e} de Lille, CNRS, Inria, \\ Centrale Lille, UMR 9189 -- CRIStAL, France\\}
    \and
    \IEEEauthorblockN{Abderrahmane Seriai$^2$, Laurent Deruelle$^2$,\\ Mustapha Derras$^2$}\IEEEauthorblockA{$^2$Berger-Levrault, France}     
}

\begin{document}
\title{GUI application meta-models}


\date{\today}
\maketitle



\begin{abstract}

\bvc{In this context...}
When a developer wants to analyse an application,
    and this application includes an user interface.
He could want an abstraction of the application.
Often, this abstraction level correspond to models, and their meta-models, of
    the piece of software.
\bvc{We consider this problem P...}
There are indeed many elements to represent and to link.
\bvc{P is a problem because...}
Currently there are many meta-models of GUI application
    that can be used to represent the interface
    and some links between the different windows.
But none of them express complex behavior, such as loop or condition,
    nor data structure information.
\bvc{We propose this solution...} 
We defined four meta-models.
The first one represent the Graphical User Interface,
    the second one the layout to apply to the GUI, 
    the third one the data structure implies in the GUI, 
    the last one the behavior associate to an event fired by an element of the GUI. 
\bvc{Our solution solves P in such and such way.}
Our meta-models can express the different elements of a GUI application.
So can represent the graphical user interface
    and the logic of the application.

   
\end{abstract}

\begin{IEEEkeywords}
    Graphical User Interfaces, Model-Driven Engineering
\end{IEEEkeywords}

\section{Introduction}
\label{sec:intro}

% Contexte
% - Entreprise internationale
% - Logiciel de plus de X MLOC
% - Nombre d'écran
% - Année d'existence (RH +~ 10 ans)
% - Nombre de dev
% - GWT en perte de vitesse


% Problème


% Known tracks for \sd{solutions} here you want to show that you are not an idiot not knowing what have been around


% What our solution is \ct{Set} and \ct{OrderedCollection} (so that the reader knows where the paper is going)
% - Un outil réutilisable
% - Migrer l'ensemble du code lié à l'interface Graphique (sans code métier) -> Mesurable
% - Recherche de correspondance entre ancien et nouveau framework (GWT to HTML/PrimeNG)

% Contribution of the paper


% Paper structure


\section{Problem Description}
\label{sec:problem}


\section{Migration Solution}
\label{sec:Solution}

% Context, exposed with the \textbf{most precise terms possible} (don't open
% unwanted doors for the reader)


% Probably set the vocabulary before to cut any misinterpretation

% Constraints that influenced the solution (because the solution is not
% universal) \emph{e.g.} our requirements for a solution, possibly not all
% satisfied. They should be sound and believable. Analysis of the criteria.
% Imagine that you are another guy having this problem do the constraint
% matches yours so that you could apply the solution


% Factual solution tracks, to position...
% Our solution in a nutshell.

\section{Proposed Solution}
\label{sec:contribution}

% Free form, variable number of sections, technical details.

% But in general do not mix solution and discussions/possible variation
% let that for discussion



\section{Conclusion}
\label{sec:conclusion}

% In this paper, we \textsf{looked}\xspace at problem P with this context and these
% constraints. We proposed solution S. It has such good points and such not so
% good ones. Now we could do this or that.

 

\subsection*{Acknowledgements} 
This work was supported by Ministry of Higher Education and Research, Nord-Pas de Calais Regional Council, CPER Nord-Pas de Calais/FEDER DATA Advanced data science and technologies 2015-2020.

%\bibliographystyle{abbrv}
%natbib 
\bibliographystyle{myplainnat}
%\bibliography{references}

\end{document}